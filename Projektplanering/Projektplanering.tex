\documentclass[12pt,a4paper]{article}
\usepackage[T1]{fontenc}
\usepackage{lmodern}
\usepackage{xcolor}
\usepackage[utf8]{inputenc}
\usepackage{graphicx}
\usepackage{fullpage}
\usepackage[font=small,labelfont=bf]{caption}
\usepackage{caption}
\usepackage{multirow}
\usepackage{wrapfig}
\usepackage{float}
\usepackage{multirow}
\newcommand{\HRule}{\rule{\linewidth}{0.5mm}}
\usepackage[hidelinks]{hyperref}
\usepackage{array,etoolbox}
\preto\tabular{\setcounter{magicrownumbers}{0}}
\newcounter{magicrownumbers}
\newcommand\rownumber{\stepcounter{magicrownumbers}\arabic{magicrownumbers}}
\usepackage{tikz}
\parindent0em

\begin{document}
	\begin{titlepage}
  	\vspace*{3\baselineskip}
    \begin{center}
    	\begin{center}
    \includegraphics[scale=0.5]{Faculty_of_Engineering_(LTH),_Lund_University_logo.png}
    	\end{center}
    \vspace*{3\baselineskip}
    \large
    \bfseries
   \Huge
   Optimizing business intelligence extraction speed from an ERP-system's database
   \HRule\\
       \normalfont
          \LARGE
          Project plan \\
            \normalsize
            \vspace*{1\baselineskip}
      Master Thesis in  Computer Science\\
      \vspace*{4\baselineskip}
      Preliminary dates: 20/1/2015 --- 9/6/2015\\
      \vspace*{6\baselineskip}
      
      
      \begin{table}[H]
      \centering
      \begin{tabular}{r c l}
      Max Åberg &|& Alexander Söderberg\\
      (880408-0219) &|& (900702-5217) \\
      mat09mab@student.lu.se &|& ain10aso@student.lu.se\\
      \end{tabular}
      \end{table}

    
    

    \today \\

    \end{center}
\end{titlepage}
    
\newpage

\vspace*{-2\baselineskip}

\section*{Introduction}
PerfectIT BeX AB has three products distributed as SaaS-solutions. The three products are:
\begin{itemize}
\item BeX Online - a cloud based ERP-system specialized for retail and e-commerce businesses. BeX Online has features for finance, sales orders, purchase orders, inventory/warehousing, business intelligence-reporting and more. 
\item BeX Retail - a desktop retail application connected to BeX Online.
\item BeX B2B - a cloud based system for order handling inbetween businesses.
\end{itemize}

BeX Online has a module for creating business intelligence reports (refered to as BI-reports) but generating what is considered as large reports is slow and puts a high demand on the system. The company believes that this process could be faster and more effective. 

\section*{Goal}
PerfectIt BeX AB wants the process behind generating these reports analyzed and optimised to improve the speed of the generation and lessen the demands it puts on the system.
\section*{Stakeholders}
\begin{table}[H]
    \begin{tabular}{l|l|l}
    Stakeholder & Name & Mail \\\hline
    Thesis students & Max Åberg \& Alexander Söderberg & see frontpage \\\hline
    University supervisor & Alma Orucevic Alagic & alma@cs.lth.se\\\hline
    Workplace supervisor & Lennart Söderberg & lennart@perfectit.se\\\hline
    Examiner & Per Andersson & per.andersson@cs.lth.se 
    \end{tabular}
    \caption{Thesis stakeholders}
\end{table}

\section*{Overall objectives and issues/research questions}
The current processes behind generation of the BI-reports must be identified and mapped. This mapping will be analyzed for bottle-necks and innefficiencies. Research will be performed to learn of approaches to solve the problems identified in the analysis. The research will consider as many options as possible to find a long-term solution.
These approaches will be analyzed to decide which of them are a best fit and should be used to reach the goal. This will then be implemented in BeX Online and a final analysis should be performed to determine if the goal was met.
It is important that the new solution does not affect the front-end interface of the system.

\section*{Result list}
\begin{itemize}
\item something
\end{itemize}

\section*{Activity breakdown}
\begin{table}[H]
    \begin{tabular}{|@{\makebox[1em][r]{\rownumber\space}} l|l|l|}
    \hline
    \multicolumn{1}{|l|}{\textbf{Activities}} & \textbf{Time} & \textbf{Dependent on}\\\hline
    Literature search & 7 days & \ldots \\\hline
    literature summary & 12 days & 1\\\hline   
    \end{tabular}
    \caption{Example of activities}
\end{table}

\section*{Time schedule}

\section*{Follow up}

\end{document}