\documentclass[12pt,a4paper]{article}
\usepackage[T1]{fontenc}
\usepackage{lmodern}
\usepackage{xcolor}
\usepackage[utf8]{inputenc}
\usepackage{graphicx}
\usepackage{fullpage}
\usepackage[font=small,labelfont=bf]{caption}
\usepackage{caption}
\usepackage{multirow}
\usepackage{wrapfig}
\usepackage{float}
\usepackage{multirow}
\newcommand{\HRule}{\rule{\linewidth}{0.5mm}}
\usepackage[hidelinks]{hyperref}
\usepackage{array,etoolbox}
\usepackage{tikz}
\usepackage{pgfgantt}
\usepackage{rotating}
\usepackage[graphicx]{realboxes}



\begin{document}
	\begin{titlepage}
  	\vspace*{3\baselineskip}
    \begin{center}
    	\begin{center}
    \includegraphics[scale=0.5]{Faculty_of_Engineering_(LTH),_Lund_University_logo.png}
    	\end{center}
    \vspace*{3\baselineskip}
    \large
    \bfseries
   \Huge
   Optimizing business intelligence extraction speed from an ERP-system's database
   \HRule\\
       \normalfont
          \LARGE
          Project plan \\
            \normalsize
            \vspace*{1\baselineskip}
      Master Thesis in  Computer Science\\
      \vspace*{4\baselineskip}
      Preliminary dates: 20/1/2015 --- 9/6/2015\\
      \vspace*{4\baselineskip}
      
      
      \begin{table}[H]
      \centering
      \begin{tabular}{r c l}
      Max Åberg &|& Alexander Söderberg\\
      (880408-0219) &|& (900702-5217) \\
      mat09mab@student.lu.se &|& ain10aso@student.lu.se\\
      \end{tabular}
      \end{table}

    
    

    \today \\

    \end{center}
\end{titlepage}
    
\newpage

\vspace*{-2\baselineskip}

\section*{Introduction}
This documents purpose is to create a general project plan for the Master thesis's different parts. In this document the thesis's parts/tasks will be states as well as the approximate time for completion of the tasks. To ensure that the work doesn't differentiate to much from the project plan, milestones and follow ups are created and work as guidelines.

\section*{Goal}
PerfectIt BeX AB wants the process behind generating these reports analysed and optimised to improve the speed of the generation and lessen the demands it puts on the system.
\subsection*{Overall objectives and issues/research questions}
The current processes behind generation of the BI-reports must be identified and mapped. This mapping will be analysed for bottle-necks and inefficiencies. Research will be performed to learn of approaches to solve the problems identified in the analysis. The research will consider as many options as possible to find a long-term solution.
These approaches will be analysed to decide which of them are a best fit and should be used to reach the goal. This will then be implemented in BeX Online and a final analysis should be performed to determine if the goal was met.
It is important that the new solution does not affect the front-end interface of the system.

\section*{Stakeholders}
\begin{table}[H]
    \begin{tabular}{l|l|l}
    Stakeholder & Name & Mail \\\hline
    Thesis students & Max Åberg \& Alexander Söderberg & see frontpage \\\hline
    University supervisor & Alma Orucevic Alagic & alma@cs.lth.se\\\hline
    Workplace supervisor & Lennart Söderberg & lennart@perfectit.se\\\hline
    Technical supervisor & Niklas Lindgren & niklas@perfictit.se\\\hline
    Examiner & Per Andersson & per.andersson@cs.lth.se 
    \end{tabular}
    \caption{Thesis stakeholders}
\end{table}

\subsection*{Role description}
The university supervisor will aid in guiding and maintaining the thesis's focus and goals. The workplace supervisor reviews and validates the scope and goal of the thesis. The technical supervisor aids in all technical support and issues. The examiner reviews the final draft of the thesis. 

\section*{Result list}
\begin{itemize}
\item Database Description
\item Literature and Research
\item Solutions
\item Solution Software Requirements Specification
\item Implementation
\item Testing \& Validation
\item Result
\item Discussion
\end{itemize}

\section*{Activity breakdown}
\begin{table}[H]
    \begin{tabular}{| c | l | l | l |}
    \hline
    \# & \textbf{Activities} & \textbf{Time} & \textbf{Dependent on}\\\hline
		A1 & Setting up environment & 3 days & \\\hline
  		A2 & Database analysis and Benchmarking &  7 days & A1 \\\hline
  		A3 & Database description & 7 days & A2 \\\hline
    A4 & Literature research & 7 days & A3 \\\hline
    A5 & Literature summary & 5 days & A4 \\\hline
    A6 & Possible solutions & 14 days & A5 \\\hline
    A7 & Solution analysis & 7 days & A6 \\\hline
    A8 & Solution SRS & 5 days & A7 \\\hline
    A9 & Solution implementation & 21 days & A8 \\\hline
    A10 & Testing and Validation & 7 days & A9 \\\hline
    A11 & Result and Benchmarking & 14 days & A10 \\\hline
    A12 & Discussion & 14 days & A11 \\\hline
    A13 & Reviewing and improvements & 7 days & A12 \\\hline
    A14 & Presentation and opposition preparations & 7 days & A13 \\\hline
    A15 & Presentation & 1 day & A14\\\hline
    \end{tabular}
    \caption{Example of activities}
\end{table}

\section*{Time schedule}
\begin{ganttchart}[vgrid, hgrid]{1}{9}[H]
\gantttitle{Jan}{9}\\
\gantttitlelist{23,...,31}{1}\\
%First Group
\ganttbar{A1}{1}{3} \\
\ganttbar{A2}{4}{9}
%\ganttlink{elem0}{elem1}
\ganttlink{elem0}{elem1}
%\ganttmilestone{Milestone 1}{11}
%Second Group
\end{ganttchart}\\\\
\begin{ganttchart}[vgrid, hgrid]{1}{28} [H]
\gantttitle{Feb}{28}\\
\gantttitlelist{1,...,28}{1}\\
%First Group
\ganttbar{A2}{1}{1} \\
\ganttbar{A3}{2}{7} \\
\ganttbar{A4}{8}{14}\\
\ganttbar{A5}{15}{19}\\
\ganttbar{A6}{20}{28}
%\ganttlink{elem0}{elem1}
\ganttlink{elem0}{elem1}
\ganttlink{elem1}{elem2}
\ganttlink{elem2}{elem3}
\ganttlink{elem3}{elem4}
%\ganttmilestone{Milestone 1}{11}
%Second Group
\end{ganttchart}\\\\
\begin{ganttchart}[vgrid, hgrid]{1}{31} [H]
\gantttitle{Mar}{31}\\
\gantttitlelist{1,...,31}{1}\\
%First Group
\ganttbar{A6}{1}{5} \\
\ganttbar{A7}{6}{12} \\
\ganttbar{A8}{13}{17}\\
\ganttbar{A9}{18}{31}
%\ganttlink{elem0}{elem1}
\ganttlink{elem0}{elem1}
\ganttlink{elem1}{elem2}
\ganttlink{elem2}{elem3}
%\ganttmilestone{Milestone 1}{11}
%Second Group
\end{ganttchart}\\\\
\begin{ganttchart}[vgrid, hgrid]{1}{30}[H]
\gantttitle{Apr}{30}\\
\gantttitlelist{1,...,30}{1}\\
%First Group
\ganttbar{A9}{1}{7} \\
\ganttbar{A10}{8}{14} \\
\ganttbar{A11}{15}{28}\\
\ganttbar{A12}{29}{30}
%\ganttlink{elem0}{elem1}
\ganttlink{elem0}{elem1}
\ganttlink{elem1}{elem2}
\ganttlink{elem2}{elem3}
%\ganttmilestone{Milestone 1}{11}
%Second Group
\end{ganttchart}\\\\
\begin{ganttchart}[vgrid, hgrid]{1}{31} [H]
\gantttitle{May}{31}\\
\gantttitlelist{1,...,31}{1}\\
%First Group
\ganttbar{A12}{1}{12} \\
\ganttbar{A13}{13}{19}\\
\ganttbar{A14}{29}{31}
%\ganttlink{elem0}{elem1}
\ganttlink{elem0}{elem1}
\ganttlink{elem1}{elem2}
%\ganttmilestone{Milestone 1}{11}
%Second Group
\end{ganttchart}\\\\
\begin{ganttchart}[vgrid, hgrid]{1}{5} [H]
\gantttitle{Jun}{5}\\
\gantttitlelist{1,...,5}{1}\\
%First Group
\ganttbar{A14}{1}{4}\\
\ganttbar{A15}{5}{5}
\ganttlink{elem0}{elem1}
%\ganttlink{elem0}{elem1}
%\ganttlink{elem1}{elem2}
%\ganttmilestone{Milestone 1}{11}
%Second Group
\end{ganttchart}\\

\section*{Follow up}
Follow up will have the following activities.

\begin{itemize}
\item Regular meetings with the university supervisor will be had to obtain guidance and help with avoiding common pitfalls.
\item Meetings with the university supervisor will also be had to make sure that the thesis is on track and is of high quality.
\item Meetings with the workplace supervisor will be done to help with maintaining the correct focus and scope in the thesis. 
\item Sporadic meetings with the technical supervisor will be done to help with technical issues or specific questions about the architecture of the system.


\end{itemize}

\end{document}